%%%% Proceedings format for most of ACM conferences (with the exceptions listed below) and all ICPS volumes.
\documentclass[sigconf]{acmart}
%%%% As of March 2017, [siggraph] is no longer used. Please use sigconf (above) for SIGGRAPH conferences.

%%%% Proceedings format for SIGPLAN conferences 
% \documentclass[sigplan, anonymous, review]{acmart}

%%%% Proceedings format for SIGCHI conferences
% \documentclass[sigchi, review]{acmart}

%%%% To use the SIGCHI extended abstract template, please visit
% https://www.overleaf.com/read/zzzfqvkmrfzn


\usepackage{booktabs} % For formal tables
\usepackage{indentfirst}

% Copyright 
% I don't know what our copyright is on this?
%\setcopyright{none}
%\setcopyright{acmcopyright}
%\setcopyright{acmlicensed}
\setcopyright{rightsretained}
%\setcopyright{usgov}
%\setcopyright{usgovmixed}
%\setcopyright{cagov}
%\setcopyright{cagovmixed}


% DOI
%\acmDOI{10.475/123_4}

% ISBN
%\acmISBN{123-4567-24-567/08/06}

%Conference
%\acmConference[WOODSTOCK'97]{ACM Woodstock conference}{July 1997}{El
  %Paso, Texas USA}
%\acmYear{1997}
%\copyrightyear{2016}


%\acmArticle{4}
%\acmPrice{15.00}

\begin{document}
\title{Temp Title: Non-Functional Bugs Project}
%\titlenote{Produces the permission block, and
  %copyright information}
%\subtitle{Extended Abstract}
%\subtitlenote{The full version of the author's guide is available as
 % \texttt{acmart.pdf} document}


\author{Aida Radu}
\affiliation{%
  \institution{Dept. of Computing Science \\ University of Alberta}
  \city{Edmonton}
  \state{Canada}
}
\email{aradu@ualberta.ca}

\author{Sarah Nadi}
\affiliation{%
  \institution{Dept. of Computing Science \\ University of Alberta}
  \city{Edmonton}
  \country{Canada}
}
\email{nadi@ualberta.ca}


\begin{abstract}
[PLACEHOLDER TEXT]

\end{abstract}

%
% The code below should be generated by the tool at
% http://dl.acm.org/ccs.cfm
% Please copy and paste the code instead of the example below.

\begin{CCSXML}
	<ccs2012>
	<concept>
	<concept_id>10011007</concept_id>
	<concept_desc>Software and its engineering</concept_desc>
	<concept_significance>500</concept_significance>
	</concept>
    
	<concept>
	<concept_id>10011007.10010940.10011003</concept_id>
	<concept_desc>Software and its engineering~Extra-functional properties</concept_desc>
	<concept_significance>500</concept_significance>
	</concept>
	</ccs2012>
\end{CCSXML}

\ccsdesc[500]{Software and its engineering}
\ccsdesc[500]{Software and its engineering~Extra-functional properties}


\keywords{Repository Mining; Bug Detection}


\maketitle

%\input{samplebody-conf}
\section{Introduction}
\section{Background}
\section{Methodology}


To create our dataset, we looked at a variety of Github repositories written in either java or python. 


\subsection {Project selection}
We used two main methods to identify cadidate repositories for mining. 
First, we filtered Github projects to java and python repositories, and chose repos in order of the highest number of stars. We found several instances where developers corrected memory leaks, or replaced one API with another to improve performance. Thus, for our second method we identified candidate repos using Github`s search bar find similar changes. We searched for commits containing keywords such as ``Stringbuilder replace,`` ``foreach loop replace,`` and ``fix memory leak.``


\subsection{Obtaining commits}
To extract commits related to non-functional requirements, we implemented a keyword search using PyDriller \footnote{https://github.com/ishepard/pydriller, Davide Spadini 2018}. The stemmed keywords are listed in Table 1. Example word endings are in square brackets.

% ----------------------------------- WIP -------------------------------------------------
\begin{table}

  \caption{Commit Message Keywords}
  \label{tab:kwds}
\begin{tabular}{  c c c }
\toprule
 "fix"&"bug"&"error"\\
 "refactor"&"secur[ity]" &"maint[enance]"\\
 "stab[ility]"&"portab[ility]"&"efficien[cy]"\\
 "usab[ility]" & "reliab[ility]"&"testab[ility]"\\
 "changeab[ility]"& "replace"&"memory"\\
 "resource"& "runtime"&"crash"\\
 "leak" &"attack" &"authenticat[ion]"\\
 "authoriz]ation]"& "cipher"&"crack" \\ 
 "decrypt"&"encrypt"&"vulnerab[ility]"\\ 
 "minimize"&"optimize"&"slow"\\
 "\#"& &\\
\bottomrule
\end{tabular}
\end{table}

%--------------------------------------------------------------------------------------------------


For those repositories identified by star rating, we directly ran PyDriller; note that some of these projects contained zero true positives. For projects identified via the Github search, we first documented the bug that was initially found in the search, then ran the PyDriller program on the rest of the repo. Thus, these projects always had at least one true positive. 

Our PyDriller program generated a CSV file containing the id and message of the extracted commits. We then searched these files for true positives, which we documented in YAML files. 

\subsection{Documentation}
For each problem we identified, we recorded the \texttt{commit url} and \texttt{commit message}, as well as the partcular \texttt{file} and \texttt{method} where developers made the fix. Because a user`s commit message is not always thorough, we explained our interpretation of the problem under a \texttt{description} field. For fixes related to specific APIs (ex. using a Stingbuilder in place of  a String), we recorded the \texttt{API}, as well as a \texttt{rule} corresponding to the change. For commits not connected to a particular API, we created a general \texttt{suggestion} to address similar problems.

\section{Dataset}
\section{Discussion and Directions for Future Work}
\section{Conclusions}
\section*{threats to validity}


\section*{Acknowledgements}


\bibliographystyle{ACM-Reference-Format}
\bibliography{bibRaduNadi2018}

\end{document}

